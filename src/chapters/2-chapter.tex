% ----------------------------------------------------------
\chapter{Literature Review}\label{cap:literatureReview}
% ----------------------------------------------------------
Deve-se inserir texto entre as seções.

% ----------------------------------------------------------
\section{Exposição do tema ou matéria}
% ----------------------------------------------------------

Section...

% ----------------------------------------------------------
\subsection{Formatação do texto}
% ----------------------------------------------------------

Subsection...

% ----------------------------------------------------------
\subsubsection{As ilustrações}
% ----------------------------------------------------------

Subsubsection...

% ----------------------------------------------------------
\subsubsubsection{Exemplo tabela}
% ----------------------------------------------------------

Subsubsubsection


% ----------------------------------------------------------
\section{Microstructure, defects and Mechanical Properties}
% ----------------------------------------------------------

%


% ----------------------------------------------------------
\section{Miniaturized Testing}
% ----------------------------------------------------------


% 
\subsubsubsection{Relevance of testing miniaturized samples}
Testing miniaturized samples, with dimensions ranging from 0.5 mm to 10 mm, becomes
a specially relevant topic when we are interested in characterizing critical
regions of small components, or assessing the variation of properties along its 
volume.

\subsubsubsection{Limitation of}
Unlike machined components, manufactured through the extraction of excess material 
from a bulk, components manufactured by alternative processes may exhibit higher
properties variations

In these cases, it is not enough to have information on the mechanical properties
measured from samples extracted from bulk materials.



